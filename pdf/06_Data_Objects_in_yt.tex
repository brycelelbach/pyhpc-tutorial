
% Default to the notebook output style

    


% Inherit from the specified cell style.




    
\documentclass{article}

    
    
    \usepackage{graphicx} % Used to insert images
    \usepackage{adjustbox} % Used to constrain images to a maximum size 
    \usepackage{color} % Allow colors to be defined
    \usepackage{enumerate} % Needed for markdown enumerations to work
    \usepackage{geometry} % Used to adjust the document margins
    \usepackage{amsmath} % Equations
    \usepackage{amssymb} % Equations
    \usepackage[mathletters]{ucs} % Extended unicode (utf-8) support
    \usepackage[utf8x]{inputenc} % Allow utf-8 characters in the tex document
    \usepackage{fancyvrb} % verbatim replacement that allows latex
    \usepackage{grffile} % extends the file name processing of package graphics 
                         % to support a larger range 
    % The hyperref package gives us a pdf with properly built
    % internal navigation ('pdf bookmarks' for the table of contents,
    % internal cross-reference links, web links for URLs, etc.)
    \usepackage{hyperref}
    \usepackage{longtable} % longtable support required by pandoc >1.10
    \usepackage{booktabs}  % table support for pandoc > 1.12.2
    

    
    
    \definecolor{orange}{cmyk}{0,0.4,0.8,0.2}
    \definecolor{darkorange}{rgb}{.71,0.21,0.01}
    \definecolor{darkgreen}{rgb}{.12,.54,.11}
    \definecolor{myteal}{rgb}{.26, .44, .56}
    \definecolor{gray}{gray}{0.45}
    \definecolor{lightgray}{gray}{.95}
    \definecolor{mediumgray}{gray}{.8}
    \definecolor{inputbackground}{rgb}{.95, .95, .85}
    \definecolor{outputbackground}{rgb}{.95, .95, .95}
    \definecolor{traceback}{rgb}{1, .95, .95}
    % ansi colors
    \definecolor{red}{rgb}{.6,0,0}
    \definecolor{green}{rgb}{0,.65,0}
    \definecolor{brown}{rgb}{0.6,0.6,0}
    \definecolor{blue}{rgb}{0,.145,.698}
    \definecolor{purple}{rgb}{.698,.145,.698}
    \definecolor{cyan}{rgb}{0,.698,.698}
    \definecolor{lightgray}{gray}{0.5}
    
    % bright ansi colors
    \definecolor{darkgray}{gray}{0.25}
    \definecolor{lightred}{rgb}{1.0,0.39,0.28}
    \definecolor{lightgreen}{rgb}{0.48,0.99,0.0}
    \definecolor{lightblue}{rgb}{0.53,0.81,0.92}
    \definecolor{lightpurple}{rgb}{0.87,0.63,0.87}
    \definecolor{lightcyan}{rgb}{0.5,1.0,0.83}
    
    % commands and environments needed by pandoc snippets
    % extracted from the output of `pandoc -s`
    \DefineVerbatimEnvironment{Highlighting}{Verbatim}{commandchars=\\\{\}}
    % Add ',fontsize=\small' for more characters per line
    \newenvironment{Shaded}{}{}
    \newcommand{\KeywordTok}[1]{\textcolor[rgb]{0.00,0.44,0.13}{\textbf{{#1}}}}
    \newcommand{\DataTypeTok}[1]{\textcolor[rgb]{0.56,0.13,0.00}{{#1}}}
    \newcommand{\DecValTok}[1]{\textcolor[rgb]{0.25,0.63,0.44}{{#1}}}
    \newcommand{\BaseNTok}[1]{\textcolor[rgb]{0.25,0.63,0.44}{{#1}}}
    \newcommand{\FloatTok}[1]{\textcolor[rgb]{0.25,0.63,0.44}{{#1}}}
    \newcommand{\CharTok}[1]{\textcolor[rgb]{0.25,0.44,0.63}{{#1}}}
    \newcommand{\StringTok}[1]{\textcolor[rgb]{0.25,0.44,0.63}{{#1}}}
    \newcommand{\CommentTok}[1]{\textcolor[rgb]{0.38,0.63,0.69}{\textit{{#1}}}}
    \newcommand{\OtherTok}[1]{\textcolor[rgb]{0.00,0.44,0.13}{{#1}}}
    \newcommand{\AlertTok}[1]{\textcolor[rgb]{1.00,0.00,0.00}{\textbf{{#1}}}}
    \newcommand{\FunctionTok}[1]{\textcolor[rgb]{0.02,0.16,0.49}{{#1}}}
    \newcommand{\RegionMarkerTok}[1]{{#1}}
    \newcommand{\ErrorTok}[1]{\textcolor[rgb]{1.00,0.00,0.00}{\textbf{{#1}}}}
    \newcommand{\NormalTok}[1]{{#1}}
    
    % Define a nice break command that doesn't care if a line doesn't already
    % exist.
    \def\br{\hspace*{\fill} \\* }
    % Math Jax compatability definitions
    \def\gt{>}
    \def\lt{<}
    % Document parameters
    \title{06\_Data\_Objects\_in\_yt}
    
    
    

    % Pygments definitions
    
\makeatletter
\def\PY@reset{\let\PY@it=\relax \let\PY@bf=\relax%
    \let\PY@ul=\relax \let\PY@tc=\relax%
    \let\PY@bc=\relax \let\PY@ff=\relax}
\def\PY@tok#1{\csname PY@tok@#1\endcsname}
\def\PY@toks#1+{\ifx\relax#1\empty\else%
    \PY@tok{#1}\expandafter\PY@toks\fi}
\def\PY@do#1{\PY@bc{\PY@tc{\PY@ul{%
    \PY@it{\PY@bf{\PY@ff{#1}}}}}}}
\def\PY#1#2{\PY@reset\PY@toks#1+\relax+\PY@do{#2}}

\expandafter\def\csname PY@tok@gd\endcsname{\def\PY@tc##1{\textcolor[rgb]{0.63,0.00,0.00}{##1}}}
\expandafter\def\csname PY@tok@gu\endcsname{\let\PY@bf=\textbf\def\PY@tc##1{\textcolor[rgb]{0.50,0.00,0.50}{##1}}}
\expandafter\def\csname PY@tok@gt\endcsname{\def\PY@tc##1{\textcolor[rgb]{0.00,0.27,0.87}{##1}}}
\expandafter\def\csname PY@tok@gs\endcsname{\let\PY@bf=\textbf}
\expandafter\def\csname PY@tok@gr\endcsname{\def\PY@tc##1{\textcolor[rgb]{1.00,0.00,0.00}{##1}}}
\expandafter\def\csname PY@tok@cm\endcsname{\let\PY@it=\textit\def\PY@tc##1{\textcolor[rgb]{0.25,0.50,0.50}{##1}}}
\expandafter\def\csname PY@tok@vg\endcsname{\def\PY@tc##1{\textcolor[rgb]{0.10,0.09,0.49}{##1}}}
\expandafter\def\csname PY@tok@m\endcsname{\def\PY@tc##1{\textcolor[rgb]{0.40,0.40,0.40}{##1}}}
\expandafter\def\csname PY@tok@mh\endcsname{\def\PY@tc##1{\textcolor[rgb]{0.40,0.40,0.40}{##1}}}
\expandafter\def\csname PY@tok@go\endcsname{\def\PY@tc##1{\textcolor[rgb]{0.53,0.53,0.53}{##1}}}
\expandafter\def\csname PY@tok@ge\endcsname{\let\PY@it=\textit}
\expandafter\def\csname PY@tok@vc\endcsname{\def\PY@tc##1{\textcolor[rgb]{0.10,0.09,0.49}{##1}}}
\expandafter\def\csname PY@tok@il\endcsname{\def\PY@tc##1{\textcolor[rgb]{0.40,0.40,0.40}{##1}}}
\expandafter\def\csname PY@tok@cs\endcsname{\let\PY@it=\textit\def\PY@tc##1{\textcolor[rgb]{0.25,0.50,0.50}{##1}}}
\expandafter\def\csname PY@tok@cp\endcsname{\def\PY@tc##1{\textcolor[rgb]{0.74,0.48,0.00}{##1}}}
\expandafter\def\csname PY@tok@gi\endcsname{\def\PY@tc##1{\textcolor[rgb]{0.00,0.63,0.00}{##1}}}
\expandafter\def\csname PY@tok@gh\endcsname{\let\PY@bf=\textbf\def\PY@tc##1{\textcolor[rgb]{0.00,0.00,0.50}{##1}}}
\expandafter\def\csname PY@tok@ni\endcsname{\let\PY@bf=\textbf\def\PY@tc##1{\textcolor[rgb]{0.60,0.60,0.60}{##1}}}
\expandafter\def\csname PY@tok@nl\endcsname{\def\PY@tc##1{\textcolor[rgb]{0.63,0.63,0.00}{##1}}}
\expandafter\def\csname PY@tok@nn\endcsname{\let\PY@bf=\textbf\def\PY@tc##1{\textcolor[rgb]{0.00,0.00,1.00}{##1}}}
\expandafter\def\csname PY@tok@no\endcsname{\def\PY@tc##1{\textcolor[rgb]{0.53,0.00,0.00}{##1}}}
\expandafter\def\csname PY@tok@na\endcsname{\def\PY@tc##1{\textcolor[rgb]{0.49,0.56,0.16}{##1}}}
\expandafter\def\csname PY@tok@nb\endcsname{\def\PY@tc##1{\textcolor[rgb]{0.00,0.50,0.00}{##1}}}
\expandafter\def\csname PY@tok@nc\endcsname{\let\PY@bf=\textbf\def\PY@tc##1{\textcolor[rgb]{0.00,0.00,1.00}{##1}}}
\expandafter\def\csname PY@tok@nd\endcsname{\def\PY@tc##1{\textcolor[rgb]{0.67,0.13,1.00}{##1}}}
\expandafter\def\csname PY@tok@ne\endcsname{\let\PY@bf=\textbf\def\PY@tc##1{\textcolor[rgb]{0.82,0.25,0.23}{##1}}}
\expandafter\def\csname PY@tok@nf\endcsname{\def\PY@tc##1{\textcolor[rgb]{0.00,0.00,1.00}{##1}}}
\expandafter\def\csname PY@tok@si\endcsname{\let\PY@bf=\textbf\def\PY@tc##1{\textcolor[rgb]{0.73,0.40,0.53}{##1}}}
\expandafter\def\csname PY@tok@s2\endcsname{\def\PY@tc##1{\textcolor[rgb]{0.73,0.13,0.13}{##1}}}
\expandafter\def\csname PY@tok@vi\endcsname{\def\PY@tc##1{\textcolor[rgb]{0.10,0.09,0.49}{##1}}}
\expandafter\def\csname PY@tok@nt\endcsname{\let\PY@bf=\textbf\def\PY@tc##1{\textcolor[rgb]{0.00,0.50,0.00}{##1}}}
\expandafter\def\csname PY@tok@nv\endcsname{\def\PY@tc##1{\textcolor[rgb]{0.10,0.09,0.49}{##1}}}
\expandafter\def\csname PY@tok@s1\endcsname{\def\PY@tc##1{\textcolor[rgb]{0.73,0.13,0.13}{##1}}}
\expandafter\def\csname PY@tok@sh\endcsname{\def\PY@tc##1{\textcolor[rgb]{0.73,0.13,0.13}{##1}}}
\expandafter\def\csname PY@tok@sc\endcsname{\def\PY@tc##1{\textcolor[rgb]{0.73,0.13,0.13}{##1}}}
\expandafter\def\csname PY@tok@sx\endcsname{\def\PY@tc##1{\textcolor[rgb]{0.00,0.50,0.00}{##1}}}
\expandafter\def\csname PY@tok@bp\endcsname{\def\PY@tc##1{\textcolor[rgb]{0.00,0.50,0.00}{##1}}}
\expandafter\def\csname PY@tok@c1\endcsname{\let\PY@it=\textit\def\PY@tc##1{\textcolor[rgb]{0.25,0.50,0.50}{##1}}}
\expandafter\def\csname PY@tok@kc\endcsname{\let\PY@bf=\textbf\def\PY@tc##1{\textcolor[rgb]{0.00,0.50,0.00}{##1}}}
\expandafter\def\csname PY@tok@c\endcsname{\let\PY@it=\textit\def\PY@tc##1{\textcolor[rgb]{0.25,0.50,0.50}{##1}}}
\expandafter\def\csname PY@tok@mf\endcsname{\def\PY@tc##1{\textcolor[rgb]{0.40,0.40,0.40}{##1}}}
\expandafter\def\csname PY@tok@err\endcsname{\def\PY@bc##1{\setlength{\fboxsep}{0pt}\fcolorbox[rgb]{1.00,0.00,0.00}{1,1,1}{\strut ##1}}}
\expandafter\def\csname PY@tok@kd\endcsname{\let\PY@bf=\textbf\def\PY@tc##1{\textcolor[rgb]{0.00,0.50,0.00}{##1}}}
\expandafter\def\csname PY@tok@ss\endcsname{\def\PY@tc##1{\textcolor[rgb]{0.10,0.09,0.49}{##1}}}
\expandafter\def\csname PY@tok@sr\endcsname{\def\PY@tc##1{\textcolor[rgb]{0.73,0.40,0.53}{##1}}}
\expandafter\def\csname PY@tok@mo\endcsname{\def\PY@tc##1{\textcolor[rgb]{0.40,0.40,0.40}{##1}}}
\expandafter\def\csname PY@tok@kn\endcsname{\let\PY@bf=\textbf\def\PY@tc##1{\textcolor[rgb]{0.00,0.50,0.00}{##1}}}
\expandafter\def\csname PY@tok@mi\endcsname{\def\PY@tc##1{\textcolor[rgb]{0.40,0.40,0.40}{##1}}}
\expandafter\def\csname PY@tok@gp\endcsname{\let\PY@bf=\textbf\def\PY@tc##1{\textcolor[rgb]{0.00,0.00,0.50}{##1}}}
\expandafter\def\csname PY@tok@o\endcsname{\def\PY@tc##1{\textcolor[rgb]{0.40,0.40,0.40}{##1}}}
\expandafter\def\csname PY@tok@kr\endcsname{\let\PY@bf=\textbf\def\PY@tc##1{\textcolor[rgb]{0.00,0.50,0.00}{##1}}}
\expandafter\def\csname PY@tok@s\endcsname{\def\PY@tc##1{\textcolor[rgb]{0.73,0.13,0.13}{##1}}}
\expandafter\def\csname PY@tok@kp\endcsname{\def\PY@tc##1{\textcolor[rgb]{0.00,0.50,0.00}{##1}}}
\expandafter\def\csname PY@tok@w\endcsname{\def\PY@tc##1{\textcolor[rgb]{0.73,0.73,0.73}{##1}}}
\expandafter\def\csname PY@tok@kt\endcsname{\def\PY@tc##1{\textcolor[rgb]{0.69,0.00,0.25}{##1}}}
\expandafter\def\csname PY@tok@ow\endcsname{\let\PY@bf=\textbf\def\PY@tc##1{\textcolor[rgb]{0.67,0.13,1.00}{##1}}}
\expandafter\def\csname PY@tok@sb\endcsname{\def\PY@tc##1{\textcolor[rgb]{0.73,0.13,0.13}{##1}}}
\expandafter\def\csname PY@tok@k\endcsname{\let\PY@bf=\textbf\def\PY@tc##1{\textcolor[rgb]{0.00,0.50,0.00}{##1}}}
\expandafter\def\csname PY@tok@se\endcsname{\let\PY@bf=\textbf\def\PY@tc##1{\textcolor[rgb]{0.73,0.40,0.13}{##1}}}
\expandafter\def\csname PY@tok@sd\endcsname{\let\PY@it=\textit\def\PY@tc##1{\textcolor[rgb]{0.73,0.13,0.13}{##1}}}

\def\PYZbs{\char`\\}
\def\PYZus{\char`\_}
\def\PYZob{\char`\{}
\def\PYZcb{\char`\}}
\def\PYZca{\char`\^}
\def\PYZam{\char`\&}
\def\PYZlt{\char`\<}
\def\PYZgt{\char`\>}
\def\PYZsh{\char`\#}
\def\PYZpc{\char`\%}
\def\PYZdl{\char`\$}
\def\PYZhy{\char`\-}
\def\PYZsq{\char`\'}
\def\PYZdq{\char`\"}
\def\PYZti{\char`\~}
% for compatibility with earlier versions
\def\PYZat{@}
\def\PYZlb{[}
\def\PYZrb{]}
\makeatother


    % Exact colors from NB
    \definecolor{incolor}{rgb}{0.0, 0.0, 0.5}
    \definecolor{outcolor}{rgb}{0.545, 0.0, 0.0}



    
    % Prevent overflowing lines due to hard-to-break entities
    \sloppy 
    % Setup hyperref package
    \hypersetup{
      breaklinks=true,  % so long urls are correctly broken across lines
      colorlinks=true,
      urlcolor=blue,
      linkcolor=darkorange,
      citecolor=darkgreen,
      }
    % Slightly bigger margins than the latex defaults
    
    \geometry{verbose,tmargin=1in,bmargin=1in,lmargin=1in,rmargin=1in}
    
    

    \begin{document}
    
    
    \maketitle
    
    

    
    \section{Data Objects and Time Series
Data}\label{data-objects-and-time-series-data}

Just like before, we will load up yt. Since we'll be using pylab to plot
some data in this notebook, we additionally tell matplotlib to place
plots inline inside the notebook.

    \begin{Verbatim}[commandchars=\\\{\}]
{\color{incolor}In [{\color{incolor}}]:} \PY{o}{\PYZpc{}}\PY{k}{matplotlib} \PY{n}{inline}
       \PY{k+kn}{import} \PY{n+nn}{yt}
       \PY{k+kn}{import} \PY{n+nn}{numpy} \PY{k+kn}{as} \PY{n+nn}{np}
       \PY{k+kn}{from} \PY{n+nn}{matplotlib} \PY{k+kn}{import} \PY{n}{pylab}
       \PY{k+kn}{from} \PY{n+nn}{yt.analysis\PYZus{}modules.halo\PYZus{}finding.api} \PY{k+kn}{import} \PY{n}{HaloFinder}
\end{Verbatim}

    \subsection{Time Series Data}\label{time-series-data}

Unlike before, instead of loading a single dataset, this time we'll load
a bunch which we'll examine in sequence. This command creates a
\texttt{DatasetSeries} object, which can be iterated over (including in
parallel, which is outside the scope of this quickstart) and analyzed.
There are some other helpful operations it can provide, but we'll stick
to the basics here.

Note that you can specify either a list of filenames, or a glob (i.e.,
asterisk) pattern in this.

    \begin{Verbatim}[commandchars=\\\{\}]
{\color{incolor}In [{\color{incolor}}]:} \PY{n}{ts} \PY{o}{=} \PY{n}{yt}\PY{o}{.}\PY{n}{DatasetSeries}\PY{p}{(}\PY{l+s}{\PYZdq{}}\PY{l+s}{enzo\PYZus{}tiny\PYZus{}cosmology/*/*.hierarchy}\PY{l+s}{\PYZdq{}}\PY{p}{)}
\end{Verbatim}

    \subsubsection{Example 1: Simple Time
Series}\label{example-1-simple-time-series}

As a simple example of how we can use this functionality, let's find the
min and max of the density as a function of time in this simulation. To
do this we use the construction \texttt{for ds in ts} where \texttt{ds}
means ``Dataset'' and \texttt{ts} is the ``Time Series'' we just loaded
up. For each dataset, we'll create an object (\texttt{dd}) that covers
the entire domain. (\texttt{all\_data} is a shorthand function for
this.) We'll then call the \texttt{extrema} Derived Quantity, and append
the min and max to our extrema outputs.

    \begin{Verbatim}[commandchars=\\\{\}]
{\color{incolor}In [{\color{incolor}}]:} \PY{n}{rho\PYZus{}ex} \PY{o}{=} \PY{p}{[}\PY{p}{]}
       \PY{n}{times} \PY{o}{=} \PY{p}{[}\PY{p}{]}
       \PY{k}{for} \PY{n}{ds} \PY{o+ow}{in} \PY{n}{ts}\PY{p}{:}
           \PY{n}{dd} \PY{o}{=} \PY{n}{ds}\PY{o}{.}\PY{n}{all\PYZus{}data}\PY{p}{(}\PY{p}{)}
           \PY{n}{rho\PYZus{}ex}\PY{o}{.}\PY{n}{append}\PY{p}{(}\PY{n}{dd}\PY{o}{.}\PY{n}{quantities}\PY{o}{.}\PY{n}{extrema}\PY{p}{(}\PY{l+s}{\PYZdq{}}\PY{l+s}{density}\PY{l+s}{\PYZdq{}}\PY{p}{)}\PY{p}{)}
           \PY{n}{times}\PY{o}{.}\PY{n}{append}\PY{p}{(}\PY{n}{ds}\PY{o}{.}\PY{n}{current\PYZus{}time}\PY{o}{.}\PY{n}{in\PYZus{}units}\PY{p}{(}\PY{l+s}{\PYZdq{}}\PY{l+s}{Gyr}\PY{l+s}{\PYZdq{}}\PY{p}{)}\PY{p}{)}
       \PY{n}{rho\PYZus{}ex} \PY{o}{=} \PY{n}{np}\PY{o}{.}\PY{n}{array}\PY{p}{(}\PY{n}{rho\PYZus{}ex}\PY{p}{)}
\end{Verbatim}

    Now we plot the minimum and the maximum:

    \begin{Verbatim}[commandchars=\\\{\}]
{\color{incolor}In [{\color{incolor}}]:} \PY{n}{pylab}\PY{o}{.}\PY{n}{semilogy}\PY{p}{(}\PY{n}{times}\PY{p}{,} \PY{n}{rho\PYZus{}ex}\PY{p}{[}\PY{p}{:}\PY{p}{,}\PY{l+m+mi}{0}\PY{p}{]}\PY{p}{,} \PY{l+s}{\PYZsq{}}\PY{l+s}{\PYZhy{}xk}\PY{l+s}{\PYZsq{}}\PY{p}{,} \PY{n}{label}\PY{o}{=}\PY{l+s}{\PYZsq{}}\PY{l+s}{Minimum}\PY{l+s}{\PYZsq{}}\PY{p}{)}
       \PY{n}{pylab}\PY{o}{.}\PY{n}{semilogy}\PY{p}{(}\PY{n}{times}\PY{p}{,} \PY{n}{rho\PYZus{}ex}\PY{p}{[}\PY{p}{:}\PY{p}{,}\PY{l+m+mi}{1}\PY{p}{]}\PY{p}{,} \PY{l+s}{\PYZsq{}}\PY{l+s}{\PYZhy{}xr}\PY{l+s}{\PYZsq{}}\PY{p}{,} \PY{n}{label}\PY{o}{=}\PY{l+s}{\PYZsq{}}\PY{l+s}{Maximum}\PY{l+s}{\PYZsq{}}\PY{p}{)}
       \PY{n}{pylab}\PY{o}{.}\PY{n}{ylabel}\PY{p}{(}\PY{l+s}{\PYZdq{}}\PY{l+s}{Density (\PYZdl{}g/cm\PYZca{}3\PYZdl{})}\PY{l+s}{\PYZdq{}}\PY{p}{)}
       \PY{n}{pylab}\PY{o}{.}\PY{n}{xlabel}\PY{p}{(}\PY{l+s}{\PYZdq{}}\PY{l+s}{Time (Gyr)}\PY{l+s}{\PYZdq{}}\PY{p}{)}
       \PY{n}{pylab}\PY{o}{.}\PY{n}{legend}\PY{p}{(}\PY{p}{)}
       \PY{n}{pylab}\PY{o}{.}\PY{n}{ylim}\PY{p}{(}\PY{l+m+mf}{1e\PYZhy{}32}\PY{p}{,} \PY{l+m+mf}{1e\PYZhy{}21}\PY{p}{)}
       \PY{n}{pylab}\PY{o}{.}\PY{n}{show}\PY{p}{(}\PY{p}{)}
\end{Verbatim}

    \subsubsection{Example 2: Advanced Time
Series}\label{example-2-advanced-time-series}

Let's do something a bit different. Let's calculate the total mass
inside halos and outside halos.

This actually touches a lot of different pieces of machinery in yt. For
every dataset, we will run the halo finder HOP. Then, we calculate the
total mass in the domain. Then, for each halo, we calculate the sum of
the baryon mass in that halo. We'll keep running tallies of these two
things.

    \begin{Verbatim}[commandchars=\\\{\}]
{\color{incolor}In [{\color{incolor}}]:} \PY{k+kn}{from} \PY{n+nn}{yt.units} \PY{k+kn}{import} \PY{n}{Msun}
       
       \PY{n}{mass} \PY{o}{=} \PY{p}{[}\PY{p}{]}
       \PY{n}{zs} \PY{o}{=} \PY{p}{[}\PY{p}{]}
       \PY{k}{for} \PY{n}{ds} \PY{o+ow}{in} \PY{n}{ts}\PY{p}{:}
           \PY{n}{halos} \PY{o}{=} \PY{n}{HaloFinder}\PY{p}{(}\PY{n}{ds}\PY{p}{)}
           \PY{n}{dd} \PY{o}{=} \PY{n}{ds}\PY{o}{.}\PY{n}{all\PYZus{}data}\PY{p}{(}\PY{p}{)}
           \PY{n}{total\PYZus{}mass} \PY{o}{=} \PY{n}{dd}\PY{o}{.}\PY{n}{quantities}\PY{o}{.}\PY{n}{total\PYZus{}quantity}\PY{p}{(}\PY{l+s}{\PYZdq{}}\PY{l+s}{cell\PYZus{}mass}\PY{l+s}{\PYZdq{}}\PY{p}{)}\PY{o}{.}\PY{n}{in\PYZus{}units}\PY{p}{(}\PY{l+s}{\PYZdq{}}\PY{l+s}{Msun}\PY{l+s}{\PYZdq{}}\PY{p}{)}
           \PY{n}{total\PYZus{}in\PYZus{}baryons} \PY{o}{=} \PY{l+m+mf}{0.0}\PY{o}{*}\PY{n}{Msun}
           \PY{k}{for} \PY{n}{halo} \PY{o+ow}{in} \PY{n}{halos}\PY{p}{:}
               \PY{n}{sp} \PY{o}{=} \PY{n}{halo}\PY{o}{.}\PY{n}{get\PYZus{}sphere}\PY{p}{(}\PY{p}{)}
               \PY{n}{total\PYZus{}in\PYZus{}baryons} \PY{o}{+}\PY{o}{=} \PY{n}{sp}\PY{o}{.}\PY{n}{quantities}\PY{o}{.}\PY{n}{total\PYZus{}quantity}\PY{p}{(}\PY{l+s}{\PYZdq{}}\PY{l+s}{cell\PYZus{}mass}\PY{l+s}{\PYZdq{}}\PY{p}{)}\PY{o}{.}\PY{n}{in\PYZus{}units}\PY{p}{(}\PY{l+s}{\PYZdq{}}\PY{l+s}{Msun}\PY{l+s}{\PYZdq{}}\PY{p}{)}
           \PY{n}{mass}\PY{o}{.}\PY{n}{append}\PY{p}{(}\PY{n}{total\PYZus{}in\PYZus{}baryons}\PY{o}{/}\PY{n}{total\PYZus{}mass}\PY{p}{)}
           \PY{n}{zs}\PY{o}{.}\PY{n}{append}\PY{p}{(}\PY{n}{ds}\PY{o}{.}\PY{n}{current\PYZus{}redshift}\PY{p}{)}
\end{Verbatim}

    Now let's plot them!

    \begin{Verbatim}[commandchars=\\\{\}]
{\color{incolor}In [{\color{incolor}}]:} \PY{n}{pylab}\PY{o}{.}\PY{n}{semilogx}\PY{p}{(}\PY{n}{zs}\PY{p}{,} \PY{n}{mass}\PY{p}{,} \PY{l+s}{\PYZsq{}}\PY{l+s}{\PYZhy{}xb}\PY{l+s}{\PYZsq{}}\PY{p}{)}
       \PY{n}{pylab}\PY{o}{.}\PY{n}{xlabel}\PY{p}{(}\PY{l+s}{\PYZdq{}}\PY{l+s}{Redshift}\PY{l+s}{\PYZdq{}}\PY{p}{)}
       \PY{n}{pylab}\PY{o}{.}\PY{n}{ylabel}\PY{p}{(}\PY{l+s}{\PYZdq{}}\PY{l+s}{Mass in halos / Total mass}\PY{l+s}{\PYZdq{}}\PY{p}{)}
       \PY{n}{pylab}\PY{o}{.}\PY{n}{xlim}\PY{p}{(}\PY{n+nb}{max}\PY{p}{(}\PY{n}{zs}\PY{p}{)}\PY{p}{,} \PY{n+nb}{min}\PY{p}{(}\PY{n}{zs}\PY{p}{)}\PY{p}{)}
       \PY{n}{pylab}\PY{o}{.}\PY{n}{ylim}\PY{p}{(}\PY{o}{\PYZhy{}}\PY{l+m+mf}{0.01}\PY{p}{,} \PY{o}{.}\PY{l+m+mi}{18}\PY{p}{)}
\end{Verbatim}

    \subsection{Data Objects}\label{data-objects}

Time series data have many applications, but most of them rely on
examining the underlying data in some way. Below, we'll see how to use
and manipulate data objects.

\subsubsection{Ray Queries}\label{ray-queries}

yt provides the ability to examine rays, or lines, through the domain.
Note that these are not periodic, unlike most other data objects. We
create a ray object and can then examine quantities of it. Rays have the
special fields \texttt{t} and \texttt{dts}, which correspond to the time
the ray enters a given cell and the distance it travels through that
cell.

To create a ray, we specify the start and end points.

Note that we need to convert these arrays to numpy arrays due to a bug
in matplotlib 1.3.1.

    \begin{Verbatim}[commandchars=\\\{\}]
{\color{incolor}In [{\color{incolor}}]:} \PY{n}{ray} \PY{o}{=} \PY{n}{ds}\PY{o}{.}\PY{n}{ray}\PY{p}{(}\PY{p}{[}\PY{l+m+mf}{0.1}\PY{p}{,} \PY{l+m+mf}{0.2}\PY{p}{,} \PY{l+m+mf}{0.3}\PY{p}{]}\PY{p}{,} \PY{p}{[}\PY{l+m+mf}{0.9}\PY{p}{,} \PY{l+m+mf}{0.8}\PY{p}{,} \PY{l+m+mf}{0.7}\PY{p}{]}\PY{p}{)}
       \PY{n}{pylab}\PY{o}{.}\PY{n}{semilogy}\PY{p}{(}\PY{n}{np}\PY{o}{.}\PY{n}{array}\PY{p}{(}\PY{n}{ray}\PY{p}{[}\PY{l+s}{\PYZdq{}}\PY{l+s}{t}\PY{l+s}{\PYZdq{}}\PY{p}{]}\PY{p}{)}\PY{p}{,} \PY{n}{np}\PY{o}{.}\PY{n}{array}\PY{p}{(}\PY{n}{ray}\PY{p}{[}\PY{l+s}{\PYZdq{}}\PY{l+s}{density}\PY{l+s}{\PYZdq{}}\PY{p}{]}\PY{p}{)}\PY{p}{)}
\end{Verbatim}

    \begin{Verbatim}[commandchars=\\\{\}]
{\color{incolor}In [{\color{incolor}}]:} \PY{k}{print} \PY{n}{ray}\PY{p}{[}\PY{l+s}{\PYZdq{}}\PY{l+s}{dts}\PY{l+s}{\PYZdq{}}\PY{p}{]}
\end{Verbatim}

    \begin{Verbatim}[commandchars=\\\{\}]
{\color{incolor}In [{\color{incolor}}]:} \PY{k}{print} \PY{n}{ray}\PY{p}{[}\PY{l+s}{\PYZdq{}}\PY{l+s}{t}\PY{l+s}{\PYZdq{}}\PY{p}{]}
\end{Verbatim}

    \begin{Verbatim}[commandchars=\\\{\}]
{\color{incolor}In [{\color{incolor}}]:} \PY{k}{print} \PY{n}{ray}\PY{p}{[}\PY{l+s}{\PYZdq{}}\PY{l+s}{x}\PY{l+s}{\PYZdq{}}\PY{p}{]}
\end{Verbatim}

    \subsubsection{Slice Queries}\label{slice-queries}

While slices are often used for visualization, they can be useful for
other operations as well. yt regards slices as multi-resolution objects.
They are an array of cells that are not all the same size; it only
returns the cells at the highest resolution that it intersects. (This is
true for all yt data objects.) Slices and projections have the special
fields \texttt{px}, \texttt{py}, \texttt{pdx} and \texttt{pdy}, which
correspond to the coordinates and half-widths in the pixel plane.

    \begin{Verbatim}[commandchars=\\\{\}]
{\color{incolor}In [{\color{incolor}}]:} \PY{n}{ds} \PY{o}{=} \PY{n}{yt}\PY{o}{.}\PY{n}{load}\PY{p}{(}\PY{l+s}{\PYZdq{}}\PY{l+s}{IsolatedGalaxy/galaxy0030/galaxy0030}\PY{l+s}{\PYZdq{}}\PY{p}{)}
       \PY{n}{v}\PY{p}{,} \PY{n}{c} \PY{o}{=} \PY{n}{ds}\PY{o}{.}\PY{n}{find\PYZus{}max}\PY{p}{(}\PY{l+s}{\PYZdq{}}\PY{l+s}{density}\PY{l+s}{\PYZdq{}}\PY{p}{)}
       \PY{n}{sl} \PY{o}{=} \PY{n}{ds}\PY{o}{.}\PY{n}{slice}\PY{p}{(}\PY{l+m+mi}{0}\PY{p}{,} \PY{n}{c}\PY{p}{[}\PY{l+m+mi}{0}\PY{p}{]}\PY{p}{)}
       \PY{k}{print} \PY{n}{sl}\PY{p}{[}\PY{l+s}{\PYZdq{}}\PY{l+s}{index}\PY{l+s}{\PYZdq{}}\PY{p}{,} \PY{l+s}{\PYZdq{}}\PY{l+s}{x}\PY{l+s}{\PYZdq{}}\PY{p}{]}
       \PY{k}{print} \PY{n}{sl}\PY{p}{[}\PY{l+s}{\PYZdq{}}\PY{l+s}{index}\PY{l+s}{\PYZdq{}}\PY{p}{,} \PY{l+s}{\PYZdq{}}\PY{l+s}{z}\PY{l+s}{\PYZdq{}}\PY{p}{]}
       \PY{k}{print} \PY{n}{sl}\PY{p}{[}\PY{l+s}{\PYZdq{}}\PY{l+s}{pdx}\PY{l+s}{\PYZdq{}}\PY{p}{]}
       \PY{k}{print} \PY{n}{sl}\PY{p}{[}\PY{l+s}{\PYZdq{}}\PY{l+s}{gas}\PY{l+s}{\PYZdq{}}\PY{p}{,} \PY{l+s}{\PYZdq{}}\PY{l+s}{density}\PY{l+s}{\PYZdq{}}\PY{p}{]}\PY{o}{.}\PY{n}{shape}
\end{Verbatim}

    If we want to do something interesting with a \texttt{Slice}, we can
turn it into a \texttt{FixedResolutionBuffer}. This object can be
queried and will return a 2D array of values.

    \begin{Verbatim}[commandchars=\\\{\}]
{\color{incolor}In [{\color{incolor}}]:} \PY{n}{frb} \PY{o}{=} \PY{n}{sl}\PY{o}{.}\PY{n}{to\PYZus{}frb}\PY{p}{(}\PY{p}{(}\PY{l+m+mf}{50.0}\PY{p}{,} \PY{l+s}{\PYZsq{}}\PY{l+s}{kpc}\PY{l+s}{\PYZsq{}}\PY{p}{)}\PY{p}{,} \PY{l+m+mi}{1024}\PY{p}{)}
       \PY{k}{print} \PY{n}{frb}\PY{p}{[}\PY{l+s}{\PYZdq{}}\PY{l+s}{gas}\PY{l+s}{\PYZdq{}}\PY{p}{,} \PY{l+s}{\PYZdq{}}\PY{l+s}{density}\PY{l+s}{\PYZdq{}}\PY{p}{]}\PY{o}{.}\PY{n}{shape}
\end{Verbatim}

    yt provides a few functions for writing arrays to disk, particularly in
image form. Here we'll write out the log of \texttt{density}, and then
use IPython to display it back here. Note that for the most part, you
will probably want to use a \texttt{PlotWindow} for this, but in the
case that it is useful you can directly manipulate the data.

    \begin{Verbatim}[commandchars=\\\{\}]
{\color{incolor}In [{\color{incolor}}]:} \PY{n}{yt}\PY{o}{.}\PY{n}{write\PYZus{}image}\PY{p}{(}\PY{n}{np}\PY{o}{.}\PY{n}{log10}\PY{p}{(}\PY{n}{frb}\PY{p}{[}\PY{l+s}{\PYZdq{}}\PY{l+s}{gas}\PY{l+s}{\PYZdq{}}\PY{p}{,} \PY{l+s}{\PYZdq{}}\PY{l+s}{density}\PY{l+s}{\PYZdq{}}\PY{p}{]}\PY{p}{)}\PY{p}{,} \PY{l+s}{\PYZdq{}}\PY{l+s}{temp.png}\PY{l+s}{\PYZdq{}}\PY{p}{)}
       \PY{k+kn}{from} \PY{n+nn}{IPython.display} \PY{k+kn}{import} \PY{n}{Image}
       \PY{n}{Image}\PY{p}{(}\PY{n}{filename} \PY{o}{=} \PY{l+s}{\PYZdq{}}\PY{l+s}{temp.png}\PY{l+s}{\PYZdq{}}\PY{p}{)}
\end{Verbatim}

    \subsubsection{Off-Axis Slices}\label{off-axis-slices}

yt provides not only slices, but off-axis slices that are sometimes
called ``cutting planes.'' These are specified by (in order) a normal
vector and a center. Here we've set the normal vector to
\texttt{{[}0.2, 0.3, 0.5{]}} and the center to be the point of maximum
density.

We can then turn these directly into plot windows using \texttt{to\_pw}.
Note that the \texttt{to\_pw} and \texttt{to\_frb} methods are available
on slices, off-axis slices, and projections, and can be used on any of
them.

    \begin{Verbatim}[commandchars=\\\{\}]
{\color{incolor}In [{\color{incolor}}]:} \PY{n}{cp} \PY{o}{=} \PY{n}{ds}\PY{o}{.}\PY{n}{cutting}\PY{p}{(}\PY{p}{[}\PY{l+m+mf}{0.2}\PY{p}{,} \PY{l+m+mf}{0.3}\PY{p}{,} \PY{l+m+mf}{0.5}\PY{p}{]}\PY{p}{,} \PY{l+s}{\PYZdq{}}\PY{l+s}{max}\PY{l+s}{\PYZdq{}}\PY{p}{)}
       \PY{n}{pw} \PY{o}{=} \PY{n}{cp}\PY{o}{.}\PY{n}{to\PYZus{}pw}\PY{p}{(}\PY{n}{fields} \PY{o}{=} \PY{p}{[}\PY{p}{(}\PY{l+s}{\PYZdq{}}\PY{l+s}{gas}\PY{l+s}{\PYZdq{}}\PY{p}{,} \PY{l+s}{\PYZdq{}}\PY{l+s}{density}\PY{l+s}{\PYZdq{}}\PY{p}{)}\PY{p}{]}\PY{p}{)}
\end{Verbatim}

    Once we have our plot window from our cutting plane, we can show it
here.

    \begin{Verbatim}[commandchars=\\\{\}]
{\color{incolor}In [{\color{incolor}}]:} \PY{n}{pw}\PY{o}{.}\PY{n}{show}\PY{p}{(}\PY{p}{)}
\end{Verbatim}

    We can, as noted above, do the same with our slice:

    \begin{Verbatim}[commandchars=\\\{\}]
{\color{incolor}In [{\color{incolor}}]:} \PY{n}{pws} \PY{o}{=} \PY{n}{sl}\PY{o}{.}\PY{n}{to\PYZus{}pw}\PY{p}{(}\PY{n}{fields}\PY{o}{=}\PY{p}{[}\PY{l+s}{\PYZdq{}}\PY{l+s}{density}\PY{l+s}{\PYZdq{}}\PY{p}{]}\PY{p}{)}
       \PY{c}{\PYZsh{}pws.show()}
       \PY{k}{print} \PY{n}{pws}\PY{o}{.}\PY{n}{plots}\PY{o}{.}\PY{n}{keys}\PY{p}{(}\PY{p}{)}
\end{Verbatim}

    \subsubsection{Covering Grids}\label{covering-grids}

If we want to access a 3D array of data that spans multiple resolutions
in our simulation, we can use a covering grid. This will return a 3D
array of data, drawing from up to the resolution level specified when
creating the data. For example, if you create a covering grid that spans
two child grids of a single parent grid, it will fill those zones
covered by a zone of a child grid with the data from that child grid.
Where it is covered only by the parent grid, the cells from the parent
grid will be duplicated (appropriately) to fill the covering grid.

There are two different types of covering grids: unsmoothed and
smoothed. Smoothed grids will be filled through a cascading
interpolation process; they will be filled at level 0, interpolated to
level 1, filled at level 1, interpolated to level 2, filled at level 2,
etc. This will help to reduce edge effects. Unsmoothed covering grids
will not be interpolated, but rather values will be duplicated multiple
times.

Here we create an unsmoothed covering grid at level 2, with the left
edge at \texttt{{[}0.0, 0.0, 0.0{]}} and with dimensions equal to those
that would cover the entire domain at level 2. We can then ask for the
Density field, which will be a 3D array.

    \begin{Verbatim}[commandchars=\\\{\}]
{\color{incolor}In [{\color{incolor}}]:} \PY{n}{cg} \PY{o}{=} \PY{n}{ds}\PY{o}{.}\PY{n}{covering\PYZus{}grid}\PY{p}{(}\PY{l+m+mi}{2}\PY{p}{,} \PY{p}{[}\PY{l+m+mf}{0.0}\PY{p}{,} \PY{l+m+mf}{0.0}\PY{p}{,} \PY{l+m+mf}{0.0}\PY{p}{]}\PY{p}{,} \PY{n}{ds}\PY{o}{.}\PY{n}{domain\PYZus{}dimensions} \PY{o}{*} \PY{l+m+mi}{2}\PY{o}{*}\PY{o}{*}\PY{l+m+mi}{2}\PY{p}{)}
       \PY{k}{print} \PY{n}{cg}\PY{p}{[}\PY{l+s}{\PYZdq{}}\PY{l+s}{density}\PY{l+s}{\PYZdq{}}\PY{p}{]}\PY{o}{.}\PY{n}{shape}
\end{Verbatim}

    In this example, we do exactly the same thing: except we ask for a
\emph{smoothed} covering grid, which will reduce edge effects.

    \begin{Verbatim}[commandchars=\\\{\}]
{\color{incolor}In [{\color{incolor}}]:} \PY{n}{scg} \PY{o}{=} \PY{n}{ds}\PY{o}{.}\PY{n}{smoothed\PYZus{}covering\PYZus{}grid}\PY{p}{(}\PY{l+m+mi}{2}\PY{p}{,} \PY{p}{[}\PY{l+m+mf}{0.0}\PY{p}{,} \PY{l+m+mf}{0.0}\PY{p}{,} \PY{l+m+mf}{0.0}\PY{p}{]}\PY{p}{,} \PY{n}{ds}\PY{o}{.}\PY{n}{domain\PYZus{}dimensions} \PY{o}{*} \PY{l+m+mi}{2}\PY{o}{*}\PY{o}{*}\PY{l+m+mi}{2}\PY{p}{)}
       \PY{k}{print} \PY{n}{scg}\PY{p}{[}\PY{l+s}{\PYZdq{}}\PY{l+s}{density}\PY{l+s}{\PYZdq{}}\PY{p}{]}\PY{o}{.}\PY{n}{shape}
\end{Verbatim}


    % Add a bibliography block to the postdoc
    
    
    
    \end{document}
