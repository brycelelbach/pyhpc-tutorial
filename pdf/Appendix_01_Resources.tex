
% Default to the notebook output style

    


% Inherit from the specified cell style.




    
\documentclass{article}

    
    
    \usepackage{graphicx} % Used to insert images
    \usepackage{adjustbox} % Used to constrain images to a maximum size 
    \usepackage{color} % Allow colors to be defined
    \usepackage{enumerate} % Needed for markdown enumerations to work
    \usepackage{geometry} % Used to adjust the document margins
    \usepackage{amsmath} % Equations
    \usepackage{amssymb} % Equations
    \usepackage[mathletters]{ucs} % Extended unicode (utf-8) support
    \usepackage[utf8x]{inputenc} % Allow utf-8 characters in the tex document
    \usepackage{fancyvrb} % verbatim replacement that allows latex
    \usepackage{grffile} % extends the file name processing of package graphics 
                         % to support a larger range 
    % The hyperref package gives us a pdf with properly built
    % internal navigation ('pdf bookmarks' for the table of contents,
    % internal cross-reference links, web links for URLs, etc.)
    \usepackage{hyperref}
    \usepackage{longtable} % longtable support required by pandoc >1.10
    \usepackage{booktabs}  % table support for pandoc > 1.12.2
    

    
    
    \definecolor{orange}{cmyk}{0,0.4,0.8,0.2}
    \definecolor{darkorange}{rgb}{.71,0.21,0.01}
    \definecolor{darkgreen}{rgb}{.12,.54,.11}
    \definecolor{myteal}{rgb}{.26, .44, .56}
    \definecolor{gray}{gray}{0.45}
    \definecolor{lightgray}{gray}{.95}
    \definecolor{mediumgray}{gray}{.8}
    \definecolor{inputbackground}{rgb}{.95, .95, .85}
    \definecolor{outputbackground}{rgb}{.95, .95, .95}
    \definecolor{traceback}{rgb}{1, .95, .95}
    % ansi colors
    \definecolor{red}{rgb}{.6,0,0}
    \definecolor{green}{rgb}{0,.65,0}
    \definecolor{brown}{rgb}{0.6,0.6,0}
    \definecolor{blue}{rgb}{0,.145,.698}
    \definecolor{purple}{rgb}{.698,.145,.698}
    \definecolor{cyan}{rgb}{0,.698,.698}
    \definecolor{lightgray}{gray}{0.5}
    
    % bright ansi colors
    \definecolor{darkgray}{gray}{0.25}
    \definecolor{lightred}{rgb}{1.0,0.39,0.28}
    \definecolor{lightgreen}{rgb}{0.48,0.99,0.0}
    \definecolor{lightblue}{rgb}{0.53,0.81,0.92}
    \definecolor{lightpurple}{rgb}{0.87,0.63,0.87}
    \definecolor{lightcyan}{rgb}{0.5,1.0,0.83}
    
    % commands and environments needed by pandoc snippets
    % extracted from the output of `pandoc -s`
    \DefineVerbatimEnvironment{Highlighting}{Verbatim}{commandchars=\\\{\}}
    % Add ',fontsize=\small' for more characters per line
    \newenvironment{Shaded}{}{}
    \newcommand{\KeywordTok}[1]{\textcolor[rgb]{0.00,0.44,0.13}{\textbf{{#1}}}}
    \newcommand{\DataTypeTok}[1]{\textcolor[rgb]{0.56,0.13,0.00}{{#1}}}
    \newcommand{\DecValTok}[1]{\textcolor[rgb]{0.25,0.63,0.44}{{#1}}}
    \newcommand{\BaseNTok}[1]{\textcolor[rgb]{0.25,0.63,0.44}{{#1}}}
    \newcommand{\FloatTok}[1]{\textcolor[rgb]{0.25,0.63,0.44}{{#1}}}
    \newcommand{\CharTok}[1]{\textcolor[rgb]{0.25,0.44,0.63}{{#1}}}
    \newcommand{\StringTok}[1]{\textcolor[rgb]{0.25,0.44,0.63}{{#1}}}
    \newcommand{\CommentTok}[1]{\textcolor[rgb]{0.38,0.63,0.69}{\textit{{#1}}}}
    \newcommand{\OtherTok}[1]{\textcolor[rgb]{0.00,0.44,0.13}{{#1}}}
    \newcommand{\AlertTok}[1]{\textcolor[rgb]{1.00,0.00,0.00}{\textbf{{#1}}}}
    \newcommand{\FunctionTok}[1]{\textcolor[rgb]{0.02,0.16,0.49}{{#1}}}
    \newcommand{\RegionMarkerTok}[1]{{#1}}
    \newcommand{\ErrorTok}[1]{\textcolor[rgb]{1.00,0.00,0.00}{\textbf{{#1}}}}
    \newcommand{\NormalTok}[1]{{#1}}
    
    % Define a nice break command that doesn't care if a line doesn't already
    % exist.
    \def\br{\hspace*{\fill} \\* }
    % Math Jax compatability definitions
    \def\gt{>}
    \def\lt{<}
    % Document parameters
    \title{Appendix\_01\_Resources}
    
    
    

    % Pygments definitions
    
\makeatletter
\def\PY@reset{\let\PY@it=\relax \let\PY@bf=\relax%
    \let\PY@ul=\relax \let\PY@tc=\relax%
    \let\PY@bc=\relax \let\PY@ff=\relax}
\def\PY@tok#1{\csname PY@tok@#1\endcsname}
\def\PY@toks#1+{\ifx\relax#1\empty\else%
    \PY@tok{#1}\expandafter\PY@toks\fi}
\def\PY@do#1{\PY@bc{\PY@tc{\PY@ul{%
    \PY@it{\PY@bf{\PY@ff{#1}}}}}}}
\def\PY#1#2{\PY@reset\PY@toks#1+\relax+\PY@do{#2}}

\expandafter\def\csname PY@tok@gd\endcsname{\def\PY@tc##1{\textcolor[rgb]{0.63,0.00,0.00}{##1}}}
\expandafter\def\csname PY@tok@gu\endcsname{\let\PY@bf=\textbf\def\PY@tc##1{\textcolor[rgb]{0.50,0.00,0.50}{##1}}}
\expandafter\def\csname PY@tok@gt\endcsname{\def\PY@tc##1{\textcolor[rgb]{0.00,0.27,0.87}{##1}}}
\expandafter\def\csname PY@tok@gs\endcsname{\let\PY@bf=\textbf}
\expandafter\def\csname PY@tok@gr\endcsname{\def\PY@tc##1{\textcolor[rgb]{1.00,0.00,0.00}{##1}}}
\expandafter\def\csname PY@tok@cm\endcsname{\let\PY@it=\textit\def\PY@tc##1{\textcolor[rgb]{0.25,0.50,0.50}{##1}}}
\expandafter\def\csname PY@tok@vg\endcsname{\def\PY@tc##1{\textcolor[rgb]{0.10,0.09,0.49}{##1}}}
\expandafter\def\csname PY@tok@m\endcsname{\def\PY@tc##1{\textcolor[rgb]{0.40,0.40,0.40}{##1}}}
\expandafter\def\csname PY@tok@mh\endcsname{\def\PY@tc##1{\textcolor[rgb]{0.40,0.40,0.40}{##1}}}
\expandafter\def\csname PY@tok@go\endcsname{\def\PY@tc##1{\textcolor[rgb]{0.53,0.53,0.53}{##1}}}
\expandafter\def\csname PY@tok@ge\endcsname{\let\PY@it=\textit}
\expandafter\def\csname PY@tok@vc\endcsname{\def\PY@tc##1{\textcolor[rgb]{0.10,0.09,0.49}{##1}}}
\expandafter\def\csname PY@tok@il\endcsname{\def\PY@tc##1{\textcolor[rgb]{0.40,0.40,0.40}{##1}}}
\expandafter\def\csname PY@tok@cs\endcsname{\let\PY@it=\textit\def\PY@tc##1{\textcolor[rgb]{0.25,0.50,0.50}{##1}}}
\expandafter\def\csname PY@tok@cp\endcsname{\def\PY@tc##1{\textcolor[rgb]{0.74,0.48,0.00}{##1}}}
\expandafter\def\csname PY@tok@gi\endcsname{\def\PY@tc##1{\textcolor[rgb]{0.00,0.63,0.00}{##1}}}
\expandafter\def\csname PY@tok@gh\endcsname{\let\PY@bf=\textbf\def\PY@tc##1{\textcolor[rgb]{0.00,0.00,0.50}{##1}}}
\expandafter\def\csname PY@tok@ni\endcsname{\let\PY@bf=\textbf\def\PY@tc##1{\textcolor[rgb]{0.60,0.60,0.60}{##1}}}
\expandafter\def\csname PY@tok@nl\endcsname{\def\PY@tc##1{\textcolor[rgb]{0.63,0.63,0.00}{##1}}}
\expandafter\def\csname PY@tok@nn\endcsname{\let\PY@bf=\textbf\def\PY@tc##1{\textcolor[rgb]{0.00,0.00,1.00}{##1}}}
\expandafter\def\csname PY@tok@no\endcsname{\def\PY@tc##1{\textcolor[rgb]{0.53,0.00,0.00}{##1}}}
\expandafter\def\csname PY@tok@na\endcsname{\def\PY@tc##1{\textcolor[rgb]{0.49,0.56,0.16}{##1}}}
\expandafter\def\csname PY@tok@nb\endcsname{\def\PY@tc##1{\textcolor[rgb]{0.00,0.50,0.00}{##1}}}
\expandafter\def\csname PY@tok@nc\endcsname{\let\PY@bf=\textbf\def\PY@tc##1{\textcolor[rgb]{0.00,0.00,1.00}{##1}}}
\expandafter\def\csname PY@tok@nd\endcsname{\def\PY@tc##1{\textcolor[rgb]{0.67,0.13,1.00}{##1}}}
\expandafter\def\csname PY@tok@ne\endcsname{\let\PY@bf=\textbf\def\PY@tc##1{\textcolor[rgb]{0.82,0.25,0.23}{##1}}}
\expandafter\def\csname PY@tok@nf\endcsname{\def\PY@tc##1{\textcolor[rgb]{0.00,0.00,1.00}{##1}}}
\expandafter\def\csname PY@tok@si\endcsname{\let\PY@bf=\textbf\def\PY@tc##1{\textcolor[rgb]{0.73,0.40,0.53}{##1}}}
\expandafter\def\csname PY@tok@s2\endcsname{\def\PY@tc##1{\textcolor[rgb]{0.73,0.13,0.13}{##1}}}
\expandafter\def\csname PY@tok@vi\endcsname{\def\PY@tc##1{\textcolor[rgb]{0.10,0.09,0.49}{##1}}}
\expandafter\def\csname PY@tok@nt\endcsname{\let\PY@bf=\textbf\def\PY@tc##1{\textcolor[rgb]{0.00,0.50,0.00}{##1}}}
\expandafter\def\csname PY@tok@nv\endcsname{\def\PY@tc##1{\textcolor[rgb]{0.10,0.09,0.49}{##1}}}
\expandafter\def\csname PY@tok@s1\endcsname{\def\PY@tc##1{\textcolor[rgb]{0.73,0.13,0.13}{##1}}}
\expandafter\def\csname PY@tok@sh\endcsname{\def\PY@tc##1{\textcolor[rgb]{0.73,0.13,0.13}{##1}}}
\expandafter\def\csname PY@tok@sc\endcsname{\def\PY@tc##1{\textcolor[rgb]{0.73,0.13,0.13}{##1}}}
\expandafter\def\csname PY@tok@sx\endcsname{\def\PY@tc##1{\textcolor[rgb]{0.00,0.50,0.00}{##1}}}
\expandafter\def\csname PY@tok@bp\endcsname{\def\PY@tc##1{\textcolor[rgb]{0.00,0.50,0.00}{##1}}}
\expandafter\def\csname PY@tok@c1\endcsname{\let\PY@it=\textit\def\PY@tc##1{\textcolor[rgb]{0.25,0.50,0.50}{##1}}}
\expandafter\def\csname PY@tok@kc\endcsname{\let\PY@bf=\textbf\def\PY@tc##1{\textcolor[rgb]{0.00,0.50,0.00}{##1}}}
\expandafter\def\csname PY@tok@c\endcsname{\let\PY@it=\textit\def\PY@tc##1{\textcolor[rgb]{0.25,0.50,0.50}{##1}}}
\expandafter\def\csname PY@tok@mf\endcsname{\def\PY@tc##1{\textcolor[rgb]{0.40,0.40,0.40}{##1}}}
\expandafter\def\csname PY@tok@err\endcsname{\def\PY@bc##1{\setlength{\fboxsep}{0pt}\fcolorbox[rgb]{1.00,0.00,0.00}{1,1,1}{\strut ##1}}}
\expandafter\def\csname PY@tok@kd\endcsname{\let\PY@bf=\textbf\def\PY@tc##1{\textcolor[rgb]{0.00,0.50,0.00}{##1}}}
\expandafter\def\csname PY@tok@ss\endcsname{\def\PY@tc##1{\textcolor[rgb]{0.10,0.09,0.49}{##1}}}
\expandafter\def\csname PY@tok@sr\endcsname{\def\PY@tc##1{\textcolor[rgb]{0.73,0.40,0.53}{##1}}}
\expandafter\def\csname PY@tok@mo\endcsname{\def\PY@tc##1{\textcolor[rgb]{0.40,0.40,0.40}{##1}}}
\expandafter\def\csname PY@tok@kn\endcsname{\let\PY@bf=\textbf\def\PY@tc##1{\textcolor[rgb]{0.00,0.50,0.00}{##1}}}
\expandafter\def\csname PY@tok@mi\endcsname{\def\PY@tc##1{\textcolor[rgb]{0.40,0.40,0.40}{##1}}}
\expandafter\def\csname PY@tok@gp\endcsname{\let\PY@bf=\textbf\def\PY@tc##1{\textcolor[rgb]{0.00,0.00,0.50}{##1}}}
\expandafter\def\csname PY@tok@o\endcsname{\def\PY@tc##1{\textcolor[rgb]{0.40,0.40,0.40}{##1}}}
\expandafter\def\csname PY@tok@kr\endcsname{\let\PY@bf=\textbf\def\PY@tc##1{\textcolor[rgb]{0.00,0.50,0.00}{##1}}}
\expandafter\def\csname PY@tok@s\endcsname{\def\PY@tc##1{\textcolor[rgb]{0.73,0.13,0.13}{##1}}}
\expandafter\def\csname PY@tok@kp\endcsname{\def\PY@tc##1{\textcolor[rgb]{0.00,0.50,0.00}{##1}}}
\expandafter\def\csname PY@tok@w\endcsname{\def\PY@tc##1{\textcolor[rgb]{0.73,0.73,0.73}{##1}}}
\expandafter\def\csname PY@tok@kt\endcsname{\def\PY@tc##1{\textcolor[rgb]{0.69,0.00,0.25}{##1}}}
\expandafter\def\csname PY@tok@ow\endcsname{\let\PY@bf=\textbf\def\PY@tc##1{\textcolor[rgb]{0.67,0.13,1.00}{##1}}}
\expandafter\def\csname PY@tok@sb\endcsname{\def\PY@tc##1{\textcolor[rgb]{0.73,0.13,0.13}{##1}}}
\expandafter\def\csname PY@tok@k\endcsname{\let\PY@bf=\textbf\def\PY@tc##1{\textcolor[rgb]{0.00,0.50,0.00}{##1}}}
\expandafter\def\csname PY@tok@se\endcsname{\let\PY@bf=\textbf\def\PY@tc##1{\textcolor[rgb]{0.73,0.40,0.13}{##1}}}
\expandafter\def\csname PY@tok@sd\endcsname{\let\PY@it=\textit\def\PY@tc##1{\textcolor[rgb]{0.73,0.13,0.13}{##1}}}

\def\PYZbs{\char`\\}
\def\PYZus{\char`\_}
\def\PYZob{\char`\{}
\def\PYZcb{\char`\}}
\def\PYZca{\char`\^}
\def\PYZam{\char`\&}
\def\PYZlt{\char`\<}
\def\PYZgt{\char`\>}
\def\PYZsh{\char`\#}
\def\PYZpc{\char`\%}
\def\PYZdl{\char`\$}
\def\PYZhy{\char`\-}
\def\PYZsq{\char`\'}
\def\PYZdq{\char`\"}
\def\PYZti{\char`\~}
% for compatibility with earlier versions
\def\PYZat{@}
\def\PYZlb{[}
\def\PYZrb{]}
\makeatother


    % Exact colors from NB
    \definecolor{incolor}{rgb}{0.0, 0.0, 0.5}
    \definecolor{outcolor}{rgb}{0.545, 0.0, 0.0}



    
    % Prevent overflowing lines due to hard-to-break entities
    \sloppy 
    % Setup hyperref package
    \hypersetup{
      breaklinks=true,  % so long urls are correctly broken across lines
      colorlinks=true,
      urlcolor=blue,
      linkcolor=darkorange,
      citecolor=darkgreen,
      }
    % Slightly bigger margins than the latex defaults
    
    \geometry{verbose,tmargin=1in,bmargin=1in,lmargin=1in,rmargin=1in}
    
    

    \begin{document}
    
    
    \maketitle
    
    

    
    \section{Appendix 1: Resources}\label{appendix-1-resources}

\section{Python in HPC}\label{python-in-hpc}

Author: Andy Terrel

\href{http://creativecommons.org/licenses/by/3.0/deed.en_US}{\includegraphics{../figures/creative_commons_logo.png}}


    \begin{center}\rule{3in}{0.4pt}\end{center}

\subsection{Learning Python}\label{learning-python}

There are a very large number of resources for learning the python
language. The issue of course is finding a resource that is attractive
to the correct mindset. Very often something that a programmer is able
to learn from has little resonance with an artist. For this reason, I am
going to list off a couple of places that I find particularly good, but
of course I have been programming for over a decade.

While learning the language is important, the real value of Python is
the numerous libraries and the community around the tools. With that
said finding what your particular community is doing is pretty
important, but there are a core set of tools that are used by most
pythonistas. I list off a few of these tools for your consumption.

Also, because we like to see what people are doing with python, I list
off a few shining examples of python in the wide world of scientific
computing. I have to admit that there are a huge number of possibilities
but few in the realm of HPC. At the very least, a HPC person should know
Python for its scripting abilities but as the other sections of this
document underscore, Python is a good candidate for HPC codes as well.

\subsubsection{Python Tutorials}\label{python-tutorials}

\begin{itemize}
\itemsep1pt\parskip0pt\parsep0pt
\item
  \textbf{\href{http://docs.python.org/tutorial/}{Python Doc Tutorial}}:
  This is the tutorial written by the developers of Python, best for
  programmers.
\item
  \textbf{\href{http://www.greenteapress.com/thinkpython/}{Think
  Python}}: A book for non-programmers.
\end{itemize}

\subsubsection{Python Tools}\label{python-tools}

\begin{itemize}
\itemsep1pt\parskip0pt\parsep0pt
\item
  \textbf{\href{http://www.enthought.com/products/epd.php}{Enthought
  Python Distribution}}: A distribution of the most commonly used tools
  by the community (free for academics).
\item
  \textbf{\href{http://numpy.scipy.org/}{NumPy}}: Fast array library for
  Python.
\item
  \textbf{\href{http://scipy.org}{SciPy}}: A collection of scientific
  libraries.
\item
  \textbf{\href{http://matplotlib.sourceforge.net/}{MatPlotLib}}: A
  highly customizable 2D plotting library
\item
  \textbf{\href{http://ipython.org/}{IPython}}: An interactive Python
  shell and parallel code manager. The IPython notebook has become very
  popular and allows users to use an interface similar to Mathematica on
  a supercomputer.
\end{itemize}

\subsubsection{Python Stories}\label{python-stories}

\begin{itemize}
\itemsep1pt\parskip0pt\parsep0pt
\item
  \textbf{\href{http://conference.scipy.org/}{SciPy Conferences}}: The
  series of conferences associated with the scientific Python community
  see recent videos at
  \href{http://www.youtube.com/user/NextDayVideo/videos?flow=grid\&view=0}{Next
  Day Video Youtube Channel} or \href{http://pyvideo.org}{PyVideo}.
\item
  \textbf{\href{http://www.youtube.com/watch?v=mLuIB8aW2KA\&feature=youtu.be}{Python
  in Astronomy}}: Joshua Bloom from UC Berkeley gave a keynote talk at
  SciPy 2012 on ``Python as Super Glue for the Modern Scientific
  Workflow''
\item
  \textbf{\href{http://numfocus.org/user-stories/}{NumFocus User
  Stories}}: A foundation for scientific computing tools with a growing
  number of user stories.
\item
  \textbf{\href{http://numerics.kaust.edu.sa/papers/pyclaw-sisc/pyclaw-sisc.html}{PyCLAW}}:
  A petascale application written in Python
\end{itemize}

    \begin{center}\rule{3in}{0.4pt}\end{center}

\subsection{Performance}\label{performance}

Despite all the great features outlined above, the (mis)perception is
that Python is too slow for HPC Development. While it is true that
Python might not be the best language to write your tight loop and
expect a high percentage of peak flop rate, it turns out that Python has
a number of tools to help switch those lower-level languages.

To discuss performance I outline three sets of tools: profiling,
speeding up the python code via c, and speeding up python via python. It
is my view that Python has some of the best tools for looking at what
your code's performance is then drilling down to the actual bottle
necks. Speeding up code without profiling is about like trying to kill a
deer with an uzi.

\subsubsection{Python tools for
profiling}\label{python-tools-for-profiling}

\begin{itemize}
\item
  \textbf{\href{http://docs.python.org/library/profile.html}{profile and
  cProfile modules}}: These modules will give you your standard run time
  analysis and function call stack. It is pretty nice to save their
  statistics and using the pstats module you can look at the data in a
  number of ways.
\item
  \textbf{\href{http://packages.python.org/line_profiler/}{kernprof}}:
  this tool puts together many routines for doing things like line by
  line code timing
\item
  \textbf{\href{http://pypi.python.org/pypi/memory_profiler}{memory\_profiler}}:
  this tool produces line by line memory foot print of your code.
\item
  \textbf{\href{http://ipython.org/ipython-doc/dev/interactive/tutorial.html\#magic-functions}{IPython
  timers}}: The \texttt{timeit} function is quite nice for seeing the
  differences in functions in a quick interactive way.
\end{itemize}

\subsubsection{Speeding up Python}\label{speeding-up-python}

\begin{itemize}
\item
  \textbf{\href{http://cython.org/}{Cython}}: cython is the quickest way
  to take a few functions in python and get faster code. You can
  decorate the function with the cython variant of python and it
  generates c code. This is very maintable and can also link to other
  hand written code in c/c++/fortran quite easily. It is by far the
  preferred tool today.
\item
  \textbf{\href{http://docs.python.org/library/ctypes.html}{ctypes}}:
  ctypes will allow you to write your functions in c and then wrap them
  quickly with its simple decoration of the code. It handles all the
  pain of casting from PyObjects and managing the gil to call the c
  function.
\end{itemize}

Other approaches exist for writing your code in C but they are all
somewhat more for taking a C/C++ library and wrapping it in Python.

\subsubsection{Python-only approaches}\label{python-only-approaches}

If you want to stay inside Python mostly, my advice is to figure out
what data you are using and picking correct data types for implementing
your algorithms. It has been my experience that you will usually get
much farther by optimizing your data structures then any low level c
hack. For example:

\begin{itemize}
\item
  \textbf{\href{http://numpy.scipy.org/}{numpy}}: a contingous array
  very fast for strided operations of arrays
\item
  \textbf{\href{http://code.google.com/p/numexpr/}{numexpr}}: a numpy
  array expression optimizer. It allows for multithreading numpy array
  expressions and also gets rid of the numerous temporaries numpy makes
  because of restrictions of the Python interpreter.
\item
  \textbf{\href{http://pypi.python.org/pypi/blist}{blist}}: a b-tree
  implementation of a list, very fast for inserting, indexing, and
  moving the internal nodes of a list
\item
  \textbf{\href{http://pandas.pydata.org/}{pandas}}: data frames (or
  tables) very fast analytics on the arrays.
\item
  \textbf{\href{http://www.pytables.org/moin}{pytables}}: fast
  structured hierarchical tables (like hdf5), especially good for out of
  core calculations and queries to large data.
\end{itemize}

    \begin{center}\rule{3in}{0.4pt}\end{center}

\subsection{Scaling Python}\label{scaling-python}

Right now there are a few distributed Python tools but the list is
growing rapidly. Here I just give a list the tools and some domain tools
that are used in HPC that provide a Python interface.

\subsubsection{Distibuted computing
libraries}\label{distibuted-computing-libraries}

\begin{itemize}
\itemsep1pt\parskip0pt\parsep0pt
\item
  \textbf{\href{http://mpi4py.scipy.org/}{mpi4py}}: Fastest most
  complete mpi python wrapper.
\item
  \textbf{\href{http://discoproject.org/}{disco}}: Python Hadoop-like
  framework.
\item
  \textbf{\href{http://ipython.org/ipython-doc/dev/parallel/index.html}{IPython
  Parallel}}: A mpi or zero-mq based parallel python.
\item
  \textbf{\href{http://dev.danse.us/trac/pathos}{pathos}}: framework for
  heterogeneous computing
\end{itemize}

\subsubsection{Domain specific
libraries}\label{domain-specific-libraries}

\begin{itemize}
\itemsep1pt\parskip0pt\parsep0pt
\item
  \textbf{\href{http://code.google.com/p/petsc4py/}{petsc4py}}: Python
  bindings for PETSc, the Portable, Extensible Toolkit for Scientific
  Computation.
\item
  \textbf{\href{http://slepc4py.googlecode.com/}{slepc4py}}: Python
  bindings for SLEPc, the Scalable Library for Eigenvalue Problem
  Computations.
\item
  \textbf{\href{http://tao4py.googlecode.com/}{tao4py}}: Python bindings
  for TAO, the Toolkit for Advanced Optimization.
\item
  \textbf{\href{http://trilinos.sandia.gov/packages/pytrilinos/}{pyTrilinos}}:
  Trilinos wrappers
\end{itemize}


    % Add a bibliography block to the postdoc
    
    
    
    \end{document}
